\documentclass[11pt]{article}

\usepackage[top=0.5in, bottom=0.5in, left=0.5in, right=0.5in]{geometry}
\usepackage{authblk}
\usepackage{hyperref}
\usepackage[utf8]{inputenc}
\usepackage{amsmath}
\usepackage{amsfonts}
\usepackage{amssymb}
\usepackage{siunitx}
\usepackage{graphicx}
\usepackage{subcaption}
\usepackage{float}
\usepackage[nottoc,numbib]{tocbibind}
\usepackage{biblatex}

\bibliography{references.bib}

\newcommand{\email}[1]{\texttt{\href{mailto:#1}{#1}}}

\title{Brain Storm AutoMaths}
\author{brando90, kappa666\footnote{ MIT, \email{brando90, kappa666}}}

\makeatletter
\let\inserttitle\@title
\let\insertauthor\@author
\makeatother

\begin{document}

\begin{center}
  \LARGE{\inserttitle}

  \Large{\insertauthor}
\end{center}

\section{Background}

TODO

\section{Research Problem}

Automate mathematics. Automating the creativity process of forming questions and answering them. 

\section{Ideas}

\begin{enumerate}
\item generate a data set of questions of mathematics
\item we could frame it as a generation of questions. As a generative problem we could try methods that generate formal language/mathematics.
\item as a first benchmark method to compare everything that comes could be a GAN. Probably won't work great but its a benchmark. The idea is that it would learn from the data bank of mathematical question to form new mathematical questions. It would be nice if it made maths that actually made sense
\item one problem I've thought is that lots of maths includes natural language unfortunately, how to skip this issue? Is all the questions phrased essentially in a formal language?
\item whats the format for the questions? (language written, formal vs natural)
\item can we write the set of decidable problems in maths (to avoid the halting problem, that alpha go does NOT run into), for the formal/axiom system for it?
\item we can also collect the set of human relevant conjectures (cuz an arbitrary maths statement could just be a trivial conjecture...)
\item probably some relation to the sort of questions humans actually care about
\item fmri, mri brain stuff and its relations to language
\item cool links: %https://mathoverflow.net/questions/285207/is-there-research-on-human-oriented-theorem-proving,
%https://www.youtube.com/watch?v=OLxbIXwpMes&feature=youtu.be
\item how do we generate the data set? We can have UROPs generate the data set
\item it would be nice to be able to automatically generate more questions...for the models to train...how to do this in a way that preserves what humans would ask?
\item Brando thinks that probably its a good idea to focus on mathematical questions that are human like? Otherwise any arbitrary grammatically correct set of symbols of maths could be a question
\item amazon mechanical turk? for very simple maths?
\item scraping from the web?
\item scraping from online resources? UROP could code a scraper for a specific book or something (careful with copy rights)
\item for conjectures, collect all important conjectures every said?
\item for the conjecture thing, its nice cuz anything that is NOT in the set of conjectures, is a negative example, so we get lots of examples nearly for free
\item the nice thing about GANs its they generate stuff
\item reward function can be hard to define, but for maths solving its easy, if u solve the proble, success! (how do RL agents in games experience success)
\item note that the reward function for forming the questions does not seem as simple...what is the "correct answer"? When solving there is a clear success/reward!
\item usually when I do maths, the way I learn the proof (of course even though of actively doing it), I sometimes go through the proof, identify the new meaningful step(s) in the proof and add it to my intuition set of the subject (I also like what sipser does, go through the proof and do a sketch proof, compression!)
\item correct me if Im wrong, but traditional ATP don't LEARN mathematics nor form a model of INTUITION (e.g. static evluator that alpha go does)
\item would meta-learning be useful? (Ilya's AGI talk), learn to learn...maybe there is a different way to learn new maths subjects?
\end{enumerate}

\section{execution ideas}

\begin{enumerate}
\item UROP project?
\item we could collect the data set
\item Can obsidean systems help?
\item MEng thesis
\end{enumerate}

\printbibliography

\end{document}
