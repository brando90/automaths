\documentclass[11pt]{article}

\usepackage[top=0.5in, bottom=0.5in, left=0.5in, right=0.5in]{geometry}
\usepackage{authblk}
\usepackage{hyperref}
\usepackage[utf8]{inputenc}
\usepackage{amsmath}
\usepackage{amsfonts}
\usepackage{amssymb}
\usepackage{siunitx}
\usepackage{graphicx}
\usepackage{subcaption}
\usepackage{float}
\usepackage[nottoc,numbib]{tocbibind}
\usepackage{biblatex}

\bibliography{references.bib}

\newcommand{\email}[1]{\texttt{\href{mailto:#1}{#1}}}

\title{Brain Storm AutoMaths}
\author{brando90, kappa666\footnote{ MIT, \email{brando90, kappa666}}}

\makeatletter
\let\inserttitle\@title
\let\insertauthor\@author
\makeatother

\begin{document}

\begin{center}
  \LARGE{\inserttitle}

  \Large{\insertauthor}
\end{center}

\section{Background}

\section{Research Problem}

\textit{Describe the problem (either theoretical or domain specific) that will be addressed by the project.}

\section{Data}

\textit{What are the characteristics of the study area, especially those characteristics that prompted you to choose this area?
Explain the expected data to be used. If data need to be collected, before the proposal is submitted, include preliminary research exploring data availability or estimating the cost of producing these data within the period of the project.}

\begin{center}
  \begin{tabular}{ |c|c|c|c|c| }
  \hline
  \textbf{Data}  &  \textbf{Type}  &  \textbf{Year}  &  \textbf{Extent}  &  \textbf{Source}  \\
  \hline
  Google Maps   &  website  &  2017  &  Earth  &  Google Maps \cite{google_maps}  \\
  \hline
  \end{tabular}
\end{center}

\section{Methods}

\textit{How would you carry out the analysis? What kind of analysis (or analyses) do you plan to use? There should be reasonable justification of candidate methods. You may learn methods you plan to use later in the semester. The final project report should focus on this section.}

\section{Expected Results}

\textit{The expected result and conclusion that can address the question raised in the project. Expected result can include descriptive conclusion, ideally illustrated in one or few maps (graphics).}

\section{Time Line}

\textit{A preliminary schedule of each task for the project, preferably a weekly plan. It should be included in course proposal.}

\printbibliography

\end{document}